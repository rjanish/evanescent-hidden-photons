
\documentclass[prd,longbibliography,nofootinbib]{revtex4-2}
% \documentclass{article}
% \usepackage[margin=1.5in]{geometry}

% \usepackage[square,numbers]{natbib}
% \bibliographystyle{plainnat}


%fonts 
\usepackage[T1]{fontenc}
\usepackage[utf8]{inputenc}
\usepackage{relsize}
\usepackage{soul}
\usepackage{calligra}
\usepackage{verbatim}

%structure 
\usepackage{adjustbox}
\usepackage[titles]{tocloft} % table of contents style
\cftsetindents{section}{0em}{2.5em}
\cftsetindents{subsection}{2.5em}{2.5em}
\renewcommand{\cftsecfont}{\hypersetup{linkcolor=Blue}}
\renewcommand{\cftsubsecfont}{\hypersetup{linkcolor=Periwinkle}}
\renewcommand{\cftsubsubsecfont}{\hypersetup{linkcolor=Periwinkle}}
\usepackage{appendix}

%figures 
\usepackage{graphicx,epsfig}
\usepackage[usenames,dvipsnames,table]{xcolor}
\usepackage{subfig}
\usepackage{float}
\usepackage{placeins}

%tables 
\usepackage{tabularx}
\usepackage{makecell,multirow}
\usepackage{dcolumn}

%math
\usepackage{amsmath,amsthm,amssymb}
\usepackage{slashed}
\usepackage{mathrsfs}
\usepackage{bm}
\usepackage{leftidx}
\usepackage{commath}


%links
\usepackage{url}
\definecolor{red}{rgb}{0.9, 0,0}
\definecolor{navy}{rgb}{0.05, 0.05,0.8}
\usepackage[colorlinks]{hyperref}
\hypersetup{
     colorlinks   = true,
     citecolor    = red,
     linkcolor = navy
}
\graphicspath{{./figures/}}

%shorthands
\newcommand{\tx}[1]{\ensuremath{\textnormal{#1}}}
\newcommand{\eqa}[1]{\begin{align}#1\end{align}}
\newcommand{\no}{\nonumber}
\newcommand{\rp}{\right)}
\newcommand{\lp}{\left(}
\newcommand{\rb}{\right]}
\newcommand{\lb}{\left[}
\def\bit{\begin{itemize}}
\def\eit{\end{itemize}}
\def\ben{\begin{enumerate}}
\def\een{\end{enumerate}}
\newcommand{\la}[1]{\label{#1}}
\newcommand{\Eq}[1]{Eq.~\eqref{#1}} 
\newcommand{\Eqs}[2]{Eqs.~\eqref{#1} and \eqref{#2}} 
\newcommand{\Sec}[1]{Sec.~\ref{#1}} 
\newcommand{\Secs}[2]{Secs.~\ref{#1} and \ref{#2}} 
\newcommand{\Ap}[1]{Ap.~\ref{#1}} 
\newcommand{\Fig}[1]{Fig.~\ref{#1}} 
\newcommand{\Figs}[2]{Figs.~\ref{#1} and \ref{#2}}
\newcommand{\Tab}[1]{Table~\ref{#1}} 
\newcommand{\Msun}{M_\odot}
\newcommand{\Mpl}{M_\tx{Pl}}
\newcommand{\GN}{G_\tx{N}}
\newcommand{\Lqcd}{\Lambda_\tx{qcd}}
\newcommand{\mn}{\tx{min}}
\newcommand{\mx}{\tx{max}}
\newcommand{\qq}{\;\;\;\;}
\newcommand{\pfrac}[2]{\lp\frac{#1}{#2}\rp}
\newcommand{\Lag}{\mathcal{L}}
\newcommand{\tld}[1]{\tilde{#1}}

%units
\newcommand{\units}[1]{\;\tx{#1}}
\newcommand{\eV}{\units{eV}}
\newcommand{\kev}{\units{kev}}
\newcommand{\MeV}{\units{MeV}}
\newcommand{\GeV}{\units{GeV}}
\newcommand{\TeV}{\units{TeV}}
\newcommand{\Watt}{\units{W}}
\newcommand{\gram}{\units{g}}
\newcommand{\Kelvin}{\units{K}}
\newcommand{\cm}{\units{cm}}
\newcommand{\meter}{\units{m}}
\newcommand{\km}{\units{km}}
\newcommand{\Ang}{\mathop{\tx{\r{A}}}}
\newcommand{\AU}{\units{AU}}
\newcommand{\pc}{\units{pc}}
\newcommand{\kpc}{\units{kpc}}
\newcommand{\Mpc}{\units{Mpc}}
\newcommand{\seconds}{\units{sec}}
\newcommand{\yr}{\units{year}}
\newcommand{\MHz}{\units{MHz}}
\newcommand{\kHz}{\units{kHz}}
\newcommand{\Hz}{\units{Hz}}
\newcommand{\Gauss}{\units{G}}
\newcommand{\Tesla}{\units{T}}
\newcommand{\Ohm}{\Omega}
\newcommand{\Volt}{\units{V}}


%operators
% \DeclareMathOperator{\tr}{Tr}
\DeclareMathOperator{\sgn}{Sgn}
\DeclareMathOperator{\arcsinh}{Arcsinh}
\DeclareMathOperator{\arctinh}{Arctanh}
\DeclareMathOperator{\Min}{Min}
\DeclareMathOperator{\Max}{Max}
\DeclareMathOperator{\Arg}{Arg}
\DeclareMathOperator{\ReP}{Re}
\DeclareMathOperator{\ImP}{Im}
\DeclareMathOperator{\Ln}{ln}
\newcommand{\Order}{\mathcal{O}}
\newcommand{\diff}[1]{\mathop{}\!\mathrm{d}{#1}\mathop{}\!}
\newcommand{\diffp}[2]{\mathop{}\!\mathrm{d^{#1}}{#2}\mathop{}\!}
\newcommand{\dd}[2]{\frac{\partial #1}{\partial #2}}
\newcommand{\ddO}[1]{\frac{\partial}{\partial #1}}
\newcommand{\grad}{\vec{\nabla}}
\newcommand{\lvl}{\left|}
\newcommand{\rvl}{\right|}
\newcommand{\vev}[1]{\langle #1 \rangle}
\newcommand{\bra}[1]{\langle #1 | }
\newcommand{\ket}[1]{| #1 \rangle}
\newcommand{\ip}[2]{\langle #1 | #2 \rangle}
\newcommand{\ml}[3]{\langle #1 \left| #2 \right| #3 \rangle}

%formatting
\newcommand{\aside}[1]{{\color{gray} [Aside:~\emph{#1}]}}
% questions
\newcounter{questioncount}[section]
\newcommand{\question}[1]{\refstepcounter{questioncount}{\color{Maroon} [Question (\thesection.\thequestioncount): #1]}}











% annotations 
\newcounter{notecountRJ}[section]
\newcommand{\RJ}[1]{\refstepcounter{notecountRJ} {\bf \textcolor{Violet}{[RJ \thesection.\thenotecountRJ: #1]}}}

% shortands 
\newcommand{\mA}{m_{A'}}

\begin{document}

%%%%%%%%%%%%%%%%%%%%%%%%%%%%%%%%%%%%%%%%%%%%%%%% 

\title{Effervescent Oscillations of Hidden Photons}
\author{Ryan Janish}
\date{\today}

\begin{abstract}
If a massive vector field kinetically mixes with the Standard Model photon, then the static potential sourced by electromagnetic charges is not purely Coulomb in form. 
Rather, it is a sum of Coulomb and Yukawa terms. 
This spoils the ability of hollow conducting surfaces to perfectly screen static electromagnetic fields in their empty interiors. 
We show that this effect provides a technique for a hidden photon search whose mass reach is set by the thickness of a conducting shield, not the linear size of the experiment nor its source frequency. 
This allows searches to be done at larger dark photon masses than is accessible by radio frequency light-shining-through-walls experiments, while still employing sensitive low frequency electromagnetic meteorology techniques. 
A experiment comprised of a shielded resonant pick-up coil placed within the quasi-static magnetic field of a solenoid driven at low frequencies may detect hidden photons kinetic mixing parameters $\epsilon \gtrsim 10^{-13}$ and masses $10^{-4} \eV \lesssim m_{A'} \lesssim m_A' \eV$. 


\end{abstract}

\maketitle

%%%%%%%%%%%%%%%%%%%%%%%%%%%%%%%%%%%%%%%%%%%%%%%%%%%%%%%%%%%%%%%%%%%%%%%%%%%%

\section{Introduction}

\section{Effervescent Oscillation}

\subsection{Kinetic Mixing}

\subsection{Thin Parallel Plates}

\begin{figure}
\centering
\includegraphics{placeholder.pdf}
\caption{diagram of parallel plates and their various sourced field components}
\end{figure}

The failure of a conductor to screen static fields over large distances is clearly demonstrated by a simple example, one-dimensional planar example. 
Consider a uniformly charged, insulating\footnote{I.e., we take its charge density to be fixed by an external agent.} plate of infinitesimal thickness and infinite area. 
Let this plate coincide with the $xy$ plane. 
Now place a grounded, conducting plate of infinitesimal thickness and infinite area at $z=a>0$, parallel to the charged plate.  

In Maxwellian, massless electrostatics the charged plate will source a uniform electric field over $z>0$. 
The conducting plate will generate for itself a screening charge to enforce that $\vec{E}(z=a) = 0$. 
This screened charge cancels the field produced by the charged plate at $z=a$, and as byprodcut also cancels it at all $z > a$. 
The field is perfectly screened beyond the conducting plate. 

Now add a kinetically-mixed, massive hidden photon. 
The visible field sourced by the uniformly-charged plate is 
\eqa{
\label{eq:plate}
    \vec{E}_{V,\tx{source}} = 
    E_0 \lp 1 + \epsilon^2 e^{- \mA |z|} \rp
}
Here $E_0 = \sigma/2$ where $\sigma$ is the surface charge density on the charted plate, i.e.~the electric field that would be observed with no kinetic mixing. 
The field at $z=a$ is again cancelled by a screening charge induced on the conducting plate. 
By symmetry, this screening charge must be some uniform surface density $\sigma_s$ and the field it sources has the same form as \Eq{eq:plate}.
The net field for $z>a$ is 
\eqa{
\label{eq:hidden-plate}
    \vec{E}_{V} = E_0 \lp 1 + \epsilon^2 e^{- \mA z} \rp
    + \frac{\sigma_s}{2} \lp 1 + \epsilon^2 e^{- \mA \lp z - a \rp} \rp, \qq \lb z > a \rb.
}
where the conductor determines $\sigma_s$ by enforcing $\vec{E}_{V}(z=a)=0$. 
This requires  
\eqa{
    \frac{\sigma_s}{2} &= -E_0 
    \pfrac{ 1 + \epsilon^2 e^{- \mA a}}{ 1 + \epsilon^2}.
}

The field in \Eq{eq:hidden-plate} does not vanish for $z>a$ and indeed approaches a constant as $z \rightarrow \infty$. 
Consider the limit that the range of the massive photon is much shorter than the plate separation ($a \gg \mA^{-1}$) and evaluate the field far beyond conductor ($z-a \gg \mA^{-1}$). 
The screening current is 
\eqa{
    \frac{\sigma_s}{2} \approx -\frac{E_0 }{ 1 + \epsilon^2}
}
and the distant field is 
\eqa{
\label{eq:plane-leakage-field}
    \vec{E}_{V} &\approx E_0 + \frac{\sigma_s}{2} 
    \approx E_0 \lb 1 - \frac{ 1 }{ 1 + \epsilon^2} \rb  
    \approx \epsilon^2 E_0.
}
We see a non-zero leakage field at large distance from the screen, which is not suppressed by the short hidden photon range even over long distances. 

Will this effect persist for a realistic geometry? 
We have above made two idealizing assumptions for the source and shield: an infinite, perfectly planar area and an infinitesimal thickness. 
We consider both assumption in turn.
The essential behavior of \Eq{eq:plane-leakage-field}, that it is not suppressed as $\mA \rightarrow \infty$, depends on the assumption of thinness but not on that of infinite area nor planar geometry. 
The basic mechanism is that any screening charge must source a visible field whose value near the screen is enhanced by $ 1 + \Order(\epsilon^2)$ relative to its value at large distances. 
The massive component of the visible field participates in the local setting of boundary conditions on the screen, causing the screen to source a slightly smaller amplitude of the massless field that would have occurred if a massive hidden photon did not exit. 
This massless mode, which readily propagates to large distances, is thus perturbed by $\Order(\epsilon^2)$ even at those distances and perfect screening is spoiled. 
This mechanism will still operate in finite, complex geometries so long as they are larger than $\mA^{-1}$. 
We demonstrate this explicitly in \Ap{ap:finite}.  

The effect of finite thickness is discussed in \Sec{sec:thick-plate}.

Indeed $\mA$ suppression occurs only in the opposite regime, if $\mA' \rightarrow 0$ we have 
\eqa{
    \vec{E}_{V} \approx E_0 \epsilon^2 \mA^2 a z. 
}

\subsection{Thick Parallel Plates}
\label{sec:thick-plate}


\subsection{Comparison with Propagating Oscillation Experiments}

\section{Experimental Search}

\subsection{Apparatus}

\subsection{Signal Parametrics for Concentric Cylinders}

\subsection{Sensitivity}

\section{Discussion}

\bibliography{effervescent-oscillations}

\begin{appendices}

\section{Concentric Cylinders}

\section{Finite Cylinders}
\label{ap:finite}

\end{appendices}

\end{document}