
\documentclass[prd,longbibliography,nofootinbib]{revtex4-2}
% \documentclass{article}
% \usepackage[margin=1.5in]{geometry}

% \usepackage[square,numbers]{natbib}
% \bibliographystyle{plainnat}


%fonts 
\usepackage[T1]{fontenc}
\usepackage[utf8]{inputenc}
\usepackage{relsize}
\usepackage{soul}
\usepackage{calligra}
\usepackage{verbatim}

%structure 
\usepackage{adjustbox}
\usepackage[titles]{tocloft} % table of contents style
\cftsetindents{section}{0em}{2.5em}
\cftsetindents{subsection}{2.5em}{2.5em}
\renewcommand{\cftsecfont}{\hypersetup{linkcolor=Blue}}
\renewcommand{\cftsubsecfont}{\hypersetup{linkcolor=Periwinkle}}
\renewcommand{\cftsubsubsecfont}{\hypersetup{linkcolor=Periwinkle}}
\usepackage{appendix}

%figures 
\usepackage{graphicx,epsfig}
\usepackage[usenames,dvipsnames,table]{xcolor}
\usepackage{subfig}
\usepackage{float}
\usepackage{placeins}

%tables 
\usepackage{tabularx}
\usepackage{makecell,multirow}
\usepackage{dcolumn}

%math
\usepackage{amsmath,amsthm,amssymb}
\usepackage{slashed}
\usepackage{mathrsfs}
\usepackage{bm}
\usepackage{leftidx}
\usepackage{commath}


%links
\usepackage{url}
\definecolor{red}{rgb}{0.9, 0,0}
\definecolor{navy}{rgb}{0.05, 0.05,0.8}
\usepackage[colorlinks]{hyperref}
\hypersetup{
     colorlinks   = true,
     citecolor    = red,
     linkcolor = navy
}
\graphicspath{{./figures/}}

%shorthands
\newcommand{\tx}[1]{\ensuremath{\textnormal{#1}}}
\newcommand{\eqa}[1]{\begin{align}#1\end{align}}
\newcommand{\no}{\nonumber}
\newcommand{\rp}{\right)}
\newcommand{\lp}{\left(}
\newcommand{\rb}{\right]}
\newcommand{\lb}{\left[}
\def\bit{\begin{itemize}}
\def\eit{\end{itemize}}
\def\ben{\begin{enumerate}}
\def\een{\end{enumerate}}
\newcommand{\la}[1]{\label{#1}}
\newcommand{\Eq}[1]{Eq.~\eqref{#1}} 
\newcommand{\Eqs}[2]{Eqs.~\eqref{#1} and \eqref{#2}} 
\newcommand{\Sec}[1]{Sec.~\ref{#1}} 
\newcommand{\Secs}[2]{Secs.~\ref{#1} and \ref{#2}} 
\newcommand{\Ap}[1]{Ap.~\ref{#1}} 
\newcommand{\Fig}[1]{Fig.~\ref{#1}} 
\newcommand{\Figs}[2]{Figs.~\ref{#1} and \ref{#2}}
\newcommand{\Tab}[1]{Table~\ref{#1}} 
\newcommand{\Msun}{M_\odot}
\newcommand{\Mpl}{M_\tx{Pl}}
\newcommand{\GN}{G_\tx{N}}
\newcommand{\Lqcd}{\Lambda_\tx{qcd}}
\newcommand{\mn}{\tx{min}}
\newcommand{\mx}{\tx{max}}
\newcommand{\qq}{\;\;\;\;}
\newcommand{\pfrac}[2]{\lp\frac{#1}{#2}\rp}
\newcommand{\Lag}{\mathcal{L}}
\newcommand{\tld}[1]{\tilde{#1}}

%units
\newcommand{\units}[1]{\;\tx{#1}}
\newcommand{\eV}{\units{eV}}
\newcommand{\kev}{\units{kev}}
\newcommand{\MeV}{\units{MeV}}
\newcommand{\GeV}{\units{GeV}}
\newcommand{\TeV}{\units{TeV}}
\newcommand{\Watt}{\units{W}}
\newcommand{\gram}{\units{g}}
\newcommand{\Kelvin}{\units{K}}
\newcommand{\cm}{\units{cm}}
\newcommand{\meter}{\units{m}}
\newcommand{\km}{\units{km}}
\newcommand{\Ang}{\mathop{\tx{\r{A}}}}
\newcommand{\AU}{\units{AU}}
\newcommand{\pc}{\units{pc}}
\newcommand{\kpc}{\units{kpc}}
\newcommand{\Mpc}{\units{Mpc}}
\newcommand{\seconds}{\units{sec}}
\newcommand{\yr}{\units{year}}
\newcommand{\MHz}{\units{MHz}}
\newcommand{\kHz}{\units{kHz}}
\newcommand{\Hz}{\units{Hz}}
\newcommand{\Gauss}{\units{G}}
\newcommand{\Tesla}{\units{T}}
\newcommand{\Ohm}{\Omega}
\newcommand{\Volt}{\units{V}}


%operators
% \DeclareMathOperator{\tr}{Tr}
\DeclareMathOperator{\sgn}{Sgn}
\DeclareMathOperator{\arcsinh}{Arcsinh}
\DeclareMathOperator{\arctinh}{Arctanh}
\DeclareMathOperator{\Min}{Min}
\DeclareMathOperator{\Max}{Max}
\DeclareMathOperator{\Arg}{Arg}
\DeclareMathOperator{\ReP}{Re}
\DeclareMathOperator{\ImP}{Im}
\DeclareMathOperator{\Ln}{ln}
\newcommand{\Order}{\mathcal{O}}
\newcommand{\diff}[1]{\mathop{}\!\mathrm{d}{#1}\mathop{}\!}
\newcommand{\diffp}[2]{\mathop{}\!\mathrm{d^{#1}}{#2}\mathop{}\!}
\newcommand{\dd}[2]{\frac{\partial #1}{\partial #2}}
\newcommand{\ddO}[1]{\frac{\partial}{\partial #1}}
\newcommand{\grad}{\vec{\nabla}}
\newcommand{\lvl}{\left|}
\newcommand{\rvl}{\right|}
\newcommand{\vev}[1]{\langle #1 \rangle}
\newcommand{\bra}[1]{\langle #1 | }
\newcommand{\ket}[1]{| #1 \rangle}
\newcommand{\ip}[2]{\langle #1 | #2 \rangle}
\newcommand{\ml}[3]{\langle #1 \left| #2 \right| #3 \rangle}

%formatting
\newcommand{\aside}[1]{{\color{gray} [Aside:~\emph{#1}]}}
% questions
\newcounter{questioncount}[section]
\newcommand{\question}[1]{\refstepcounter{questioncount}{\color{Maroon} [Question (\thesection.\thequestioncount): #1]}}











% annotations 
\newcounter{notecountRJ}[section]
\newcommand{\RJ}[1]{\refstepcounter{notecountRJ} {\bf \textcolor{Violet}{[RJ \thesection.\thenotecountRJ: #1]}}}

% shortands 
\newcommand{\mA}{m_{A'}}
\newcommand{\vE}{\vec{E}}
\newcommand{\vB}{\vec{B}}
\newcommand{\vA}{\vec{A}}
\newcommand{\e}{\epsilon}
\newcommand{\vtE}{\tld{\vec{E}}}
\newcommand{\vtB}{\tld{\vec{B}}}

\begin{document}

%%%%%%%%%%%%%%%%%%%%%%%%%%%%%%%%%%%%%%%%%%%%%%%% 

\title{Notes on the Effervescent Mixing of Hidden Photons}
\author{Ryan Janish}
\date{\today}

\begin{abstract}
Notes for light-shining-though-walls experiments using effervescent waves with thin (super)conducing barriers and/or nested cavities. 
\end{abstract}

\maketitle

%%%%%%%%%%%%%%%%%%%%%%%%%%%%%%%%%%%%%%%%%%%%%%%%%%%%%%%%%%%%%%%%%%%%%%%%%%%%

\section{Photons and Hidden Photons}

\subsection{Kinetic Mixing}

Define the kinetic mixing $\e$ between the massless, Maxwell photon and a massive hidden photon as 
\eqa{
\la{eq:interaction-L}
    \Lag = - \frac{1}{4} F_{\mu\nu} F^{\mu\nu} 
     +  \frac{\e}{2}  F_{\mu\nu} F'^{\mu\nu}
     - \frac{1}{4}  F'_{\mu\nu} F'^{\mu\nu}
     + \frac{1}{2} \mA^2 A'_\mu A'^\mu 
     -  A_\mu J_\tx{EM}^\mu. 
} 
Here $J_\tx{EM}$ is the electromagnetic current, e.g.~$J_\tx{EM}^\mu = e \bar{\psi} \gamma^\mu \psi$ for a fermion $\psi$ of unit charge. 
As written in \Eq{eq:interaction-L}, $A$ is the active photon and $A'$ is sterile, while neither are propagation eigenstates. 
This is the interaction basis.
We can diagonalize the kinetic term with the transformation 
\eqa{
    A &= \tld{A} + \e A' .
}
This implies $ F = \tld{F} + \e F' $ and the kinetic + mixing term becomes  
\eqa{
    F^2 - 2 \e F F' + F'^2 &= 
    \lp  \tld{F}  + \e F' \rp^2 
    - 2 \e F' \lp \tld{F}  + \e F'\rp + F'^2 \\
    &= \tld{F} ^2 + 2 \e F' \tld{F}  + \e^2 F'^2
    - 2 \e F' \tld{F}  - 2 \e^2  F'^2 + F'^2 \\
    &= \tld{F} ^2 + \lp 1 -  \e^2 \rp F'^2 .
}
A direct coupling between $A'$ and the SM current appears
\eqa{
    A_\mu J_\tx{EM}^\mu = 
     \tld{A}_\mu J_\tx{EM}^\mu +  \e A'_\mu J_\tx{EM}^\mu .
}
Then, 
\eqa{
    \Lag = 
     - \frac{1}{4} \tld{F} ^2 
     - \frac{1}{4} \lp 1 -  \e^2 \rp F'^2 
     + \frac{1}{2} \mA^2 A'_\mu A'^\mu 
     -  \tld{A}_\mu J_\tx{EM}^\mu 
     -\e A'_\mu J_\tx{EM}^\mu. 
} 
Rescaling the $A'$ fields by $\sqrt{1 + \e^2}$ yields
\eqa{
\la{eq:kinetic-L}
    \Lag &= - \frac{1}{4} \tld{F}_{\mu\nu} \tld{F}^{\mu\nu}  
    -  \tld{A}_\mu J_\tx{EM}^\mu 
     - \frac{1}{4}  \tld{F}'_{\mu\nu} \tld{F}'^{\mu\nu}
     + \frac{1}{2} \frac{\mA^2}{1 - \e^2} \tld{A}'_\mu \tld{A}'^\mu 
     -  \frac{\e}{\sqrt{1 - \e^2}} \tld{A}'_\mu J_\tx{EM}^\mu. 
} 
These are the kinetic eigenstates (with tildes), related to the interacting and sterile fields (no tildes) by 
\eqa{
\la{eq:A}
   A &= \tld{A} + \frac{\e}{\sqrt{1 - \e^2}} \tld{A}' \\
\la{eq:A'}
   A' &= \frac{1}{\sqrt{1 - \e^2}} \tld{A}'
}
and
\eqa{
\la{eq:tildeA}
   \tld{A} &= A - \e A' \\
\la{eq:tildeA'}
   \tld{A}' &= \sqrt{1 - \e^2} A'
}
Note that these are not unitary transformations, so the two kinetic states $\tld{A}$ and $\tld{A}'$ are not orthogonal to each other.
That seems unusual, but I don't think it is a problem. 
$\tld{A}$ and $\tld{A}'$ are still a basis for the same field space spanned by $A$ and $A'$. 
There are of course unitary transformations that would eliminate the kinetic mixing term, such as 
\eqa{
    B &= \frac{1}{\sqrt{2}} \lp  A + A' \rp \\
    B' &= \frac{1}{\sqrt{2}} \lp  A - A' \rp
}  
however this will have a mass mixing between $B$ and $B'$ after transforming the mass term $\mA A'^2$. 
I wonder if \Eqs{eq:tildeA}{eq:tildeA'} is the unique transformation that diagonalizes both the kinetic and the mass terms, and has canonically normalized kinetic terms? 

\subsection{Field Equations}

We can infer the field equations for the active/sterile fields $\lp  E, B, E', B'\rp$ from \Eq{eq:interaction-L}.  
First replace the $F$ in the mixing term with $A$:
\eqa{
    F_{\mu\nu} F'^{\mu\nu} &= 
    \lp  \partial_\mu A_\nu - \partial_\nu A_\mu \rp F'^{\mu\nu} \\
    &= 2 \lp \partial_\mu A_\nu \rp F'^{\mu\nu} \\
    &= -2  A_\nu \lp  \partial_\mu F'^{\mu\nu} \rp 
    + \lp \tx{boundary term}\rp
\la{eq:F-to-A}
}
Thus \Eq{eq:interaction-L} can be re-written as
\eqa{
\la{eq:L-with_jeff}
    \Lag &= - \frac{1}{4} F_{\mu\nu} F^{\mu\nu} 
 - A_\nu \lp  \e \partial_\mu F'^{\mu\nu} + J_\tx{EM}^\mu \rp
     - \frac{1}{4}  F'_{\mu\nu} F'^{\mu\nu}
     + \frac{1}{2} \mA^2 A'_\mu A'^\mu 
}
and we see $F$ is sourced by the effective current $\e \partial_\mu F'^{\mu\nu} + J_\tx{EM}^\mu$. 
It thus obeys Maxwell's equations with that current, 
\eqa{
\la{eq:field-eq-F}
\partial_\mu F^{\mu\nu} =  
     \e \partial_\mu F'^{\mu\nu} + J_\tx{EM}^\nu.
}
We now make the same argument for the primed fields, replacing $F'$ with $A'$. 
By symmetry \Eq{eq:F-to-A} also yields 
\eqa{
    F_{\mu\nu} F'^{\mu\nu} 
    &= -2  A'_\nu \lp  \partial_\mu F^{\mu\nu} \rp 
    + \lp \tx{boundary term}\rp
}
and we can write \Eq{eq:interaction-L} in yet another way as
\eqa{
\la{eq:L-jeff-F'}
    \Lag &= - \frac{1}{4} F_{\mu\nu} F^{\mu\nu} 
     - A_\nu  J_\tx{EM}^\mu 
     - \frac{1}{4}  F'_{\mu\nu} F'^{\mu\nu}
     - A'_\nu \lp  - \frac{1}{2} \mA^2  A'^\nu +
     \e \partial_\mu F^{\mu\nu} \rp.
} 
So $F'$ is soured by the effective current $- \mA^2  A'^\nu + \e \partial_\mu F^{\mu\nu}$. 
Note there is an extra factor of $2$ on the $\mA$ term in the effective current compared to the Lagrangian in \Eq{eq:L-jeff-F'}. 
This is because upon varying \Eq{eq:L-jeff-F'} there is a contribution from both factors of $A'$ in the mass term. 
Thus
\eqa{
\la{eq:field-eq-F'}
    \partial_\mu F'^{\mu\nu} =
     \e \partial_\mu F^{\mu\nu} 
     - \mA^2 A'^\nu.
}

These equations describe the generation of $F$ and $F'$ fields by charges. 
To describe the response of charges to the fields, we need the Lorentz force law.
This is derived from \Eq{eq:interaction-L} by varying the term $A_\mu J_\tx{EM}^\mu$, and since that term appears here identically to that of the pure Maxwell theory, the Lorentz force here will have it usual form and involve only $F$.   
For a unit point charge with 4-velocity $U^\nu$, this is:
\eqa{
\la{eq:lorentz-force-F}
    \dd{P_\mu}{\tau} = e F^{\mu\nu} U_\nu. 
}

The field equations are coupled in $F$ and $F'$ because these field are not kinetic eigenstates. 
We can uncouple them to recover the kinetic basis, writing field equations in $\tld{F}$ and $\tld{F}'$. 
\Eq{eq:field-eq-F-vary} immediate leads us to recognize the linear combination $\tld{A} = A - \e A'$ as coupling to EM currents. 
This is the same combination identified in \Eq{eq:tildeA}. 
Applying this to \Eqs{eq:field-eq-F-vary}{eq:lorentz-force-F} yields
\eqa{
\la{eq:field-eq-tildeF}
\partial_\mu \tld{F}^{\mu\nu} &= J_\tx{EM}^\nu \\
\la{eq:lorentz-force-mixed}
    \dd{P_\mu}{\tau} &= e \tld{F}^{\mu\nu} U_\nu
     + \e e F'^{\mu\nu} U_\nu. 
}
Then combine \Eqs{eq:field-eq-F-vary}{eq:field-eq-F'-vary} to eliminate $F$, 
\eqa{
\la{eq:unscaled}
  \lp  1 - \e^2 \rp  \partial_\mu F'^{\mu\nu} 
         +  \mA^2 A'^\nu &= \e J_\tx{EM}^\nu \\
 \partial_\mu F'^{\mu\nu} 
         +  \frac{\mA^2}{1 - \e^2} A'^\nu &= 
         \frac{\e}{1 - \e^2} J_\tx{EM}^\nu.
}
Now EM currents source both $\tld{F}$ and $F'$, and they feel a force from both, and $\tld{F}$ and $F'$ do not mix. 
This is the kinetic basis. 
Note that the normalization of charge $F'$ in \Eqs{eq:unscaled}{eq:lorentz-force-mixed} is not the same as for $F$ in \Eqs{eq:field-eq-tildeF}{eq:lorentz-force-mixed}. 
$F$ couples to $J$ and to $e U$, while $F'$ couples to $\e J /(1-\e^2)$ and to $\e e U$.  
To make these have the same ratio, rescale $F'$ to be $F' = \tld{F}/\sqrt{1-\e^2}$.
This is equivalent to fixing a canonically normalized kinetic term for $\tld{F}'$. 
Then we get the equations 
\eqa{
\la{eq:field-eq-tildeF-repeate}
\partial_\mu \tld{F}^{\mu\nu} &= J_\tx{EM}^\nu \\
\la{eq:field-eq-tildeF'}
 \partial_\mu \tld{F}'^{\mu\nu} 
         +  \frac{\mA^2}{1 - \e^2} \tld{A}'^\nu &= 
         \frac{\e}{\sqrt{1 - \e^2}} J_\tx{EM}^\nu \\
    \dd{P_\mu}{\tau} &= e \tld{F}^{\mu\nu} U_\nu
     + \frac{\e}{\sqrt{1 - \e^2}} e \tld{F}'^{\mu\nu} U_\nu. 
\la{eq:lorentz-force-mixed}    
}
These three equations are precisely what we would infer from the Lagrangian in \Eq{eq:kinetic-L}. 
That is a good sanity check. 
It would probably be cleaner to first go from \Eq{eq:interaction-L} to \Eq{eq:kinetic-L}, then get the uncoupled field equations immediately from \Eq{eq:kinetic-L}, and then use \Eqs{eq:tildeA}{eq:tildeA'} to rewrite those field equations in terms of the active/sterile fields, yielding \Eqs{eq:field-eq-F}{eq:field-eq-F'}.


Now let's get equations involving explicitly the electric and magnetic fields alone. 
For reference the equations in terms of active/sterile fields are 
\eqa{
\la{eq:field-eq-F-rep}
\partial_\mu F^{\mu\nu} &=  
     \e \partial_\mu F'^{\mu\nu} + J_\tx{EM}^\nu \\
\la{eq:field-eq-F'-rep}
    \partial_\mu F'^{\mu\nu} + \mA^2 A'^\nu &=
     \e \partial_\mu F^{\mu\nu}  \\
\la{eq:lorentz-force-F-rep}
    \dd{P_\mu}{\tau} &= e F^{\mu\nu} U_\nu. 
}
Note that the usual Maxwell equations will apply to the combination $\vE - \e \vE'$ and $\vB - \e \vB'$, while the combinations $\vE' - \e \vE$ and $\vB' - \e \vB$ obey Maxwell-like equations with an effective source $-\mA^2 A'_\mu$. 
The fact that the potentials themselves appear in the field equations is expected for a massive vector.  
We have 
\eqa{
    \grad \cdot \lp  \vE - \e \vE' \rp &=  \rho_\tx{EM} \\
    \grad \times \lp \vB - \e \vB' \rp &=  
    \ddO{t} \lp  \vE - \e \vE' \rp  
    + \vec{j}_\tx{EM} \\
    \grad \cdot \lp  \vE' - \e \vE \rp &=  
     - \mA^2 \phi' \\
     \grad \times \lp \vB' - \e \vB \rp &=  
    \ddO{t} \lp  \vE' - \e \vE \rp  
     - \mA^2 \vA' 
}
and since $F$ and $F'$ are both anti-symmetric, both sets of field satisfy the source-less equations
\eqa{
    \grad \cdot \vB &= 0 \\
    \grad \times \vE &= - \dd{\vB}{t} \\
    \grad \cdot \vB' &= 0 \\
    \grad \times \vE' &= - \dd{\vB'}{t}.
}
We can thus identify two effective sources. 
The active field is sourced by 
\eqa{
    \rho_\tx{eff}^\tx{active} 
    &= \rho_\tx{EM} + \e \grad \cdot \vE' \\
    \vec{j}_\tx{eff}^\tx{active} 
    &= \vec{j}_\tx{EM} - \e \dd{\vE'}{t}
    + \e \grad \times \vB'
}
and the sterile is sourced by
\eqa{
    \rho_\tx{eff}^\tx{sterile} 
    &=  \e \grad \cdot \vE \\
    \vec{j}_\tx{eff}^\tx{sterile} 
    &= - \e \dd{\vE}{t}
    + \e \grad \times \vB
}
Note that these both satisfy $\partial_t \rho + \grad \cdot \vec{j} = 0$.
They are of course the same effected sourced identified in the Lagrangians in \Eqs{eq:L-with_jeff}{eq:L-jeff-F'}, since 
\eqa{
    \partial_\mu F^{\mu,0} &= \grad \cdot \vE \\
    \partial_\mu F^{\mu, i} &=  
    -\partial_t E_i + \lp  \grad \times \vB\rp_i 
}
And we can rewrite the field equations to be Maxwell/Proca equations with the effective sources 
\eqa{
    \grad \cdot  \vE &=  \rho_\tx{eff}^\tx{active}  \\
    \grad \times \vB - \dd{\vE}{t} &=  
     \vec{j}_\tx{eff}^\tx{active}  \\
    \grad \cdot  \vE'  + \mA^2 \phi' &=  \rho_\tx{eff}^\tx{sterile} \\
     \grad \times \vB' - \dd{\vE'}{t}    + \mA^2 \vA' &=  
     \vec{j}_\tx{eff}^\tx{sterile} .
}
We now obtain the wave equations for the electric fields in the usual manner. 
We will eliminate magnetic fields to working entirely in terms of electric fields, since in any problem the magnetic field is determined once the electric is fully known. 
Take the curl of Faraday's laws and apply the identity $\grad \times \grad \times \vec{F} = \grad \lp  \grad \cdot \vec{F} \rp - \nabla^2 \vec{F}$,
\eqa{
    \grad \lp  \grad \cdot \vE \rp - \nabla^2 \vE
     &= - \ddO{t} \lp  \grad \times \vB \rp 
}
Eliminate the magnetic field and the grad-divergence term:
\eqa{ 
    \grad \rho_\tx{eff}^\tx{active} 
    -  \nabla^2 \vE
     &= - \ddO{t} \lp  
     \dd{\vE}{t} + \vec{j}_\tx{eff}^\tx{active} \rp \\
     \lp  \partial^2_t - \nabla^2 \rp \vE &= 
- \grad \rho_\tx{eff}^\tx{active} - \partial_t \vec{j}_\tx{eff}^\tx{active} .
}
Now the effective source combination appearing here is 
\eqa{
     \grad \rho_\tx{eff}^\tx{active} + \partial_t \vec{j}_\tx{eff}^\tx{active}
     &= \grad \rho_\tx{SM} + \partial_t \vec{j}_\tx{SM}
     + \grad \lb \e \grad \cdot \vE' \rb 
+ \partial_t \lb - \e \dd{\vE'}{t}
    + \e \grad \times \vB' \rb \\
     &= \grad \rho_\tx{SM} + \partial_t \vec{j}_\tx{SM}
     +\e \lb  \grad \lp  \grad \cdot \vE' \rp 
     - \partial^2_t \vE'
    +  \partial_t \grad \times \vB' \rb 
}
and using Faraday's law
\eqa{
    \partial_t \grad \times \vB' &= 
    - \grad \times \grad \times \vE' \\
    &= - \grad \lp  \grad \cdot \vE' \rp + \nabla^2 \vE' \\
\Rightarrow \qq 
 \grad \rho_\tx{eff}^\tx{active} + \partial_t \vec{j}_\tx{eff}^\tx{active}
     &= \grad \rho_\tx{SM} + \partial_t \vec{j}_\tx{SM}
     +\e \lb  \nabla^2 \vE' 
     - \partial^2_t \vE'\rb \\
}
So the active electric field obeys 
\eqa{
    \lp  \partial^2_t - \nabla^2 \rp \vE &=  
    -\grad \rho_\tx{SM} - \partial_t \vec{j}_\tx{SM}
     +\e \lb\partial^2_t \vE' - \nabla^2 \vE' \rb.
}
Now we repeat the process for the sterile electric field. 
\eqa{
    \grad \lp  \grad \cdot \vE' \rp - \nabla^2 \vE'
     &= - \ddO{t} \lp  \grad \times \vB' \rp  \\
     \grad \lp  \rho_\tx{eff}^\tx{sterile} - \mA^2 \phi' \rp
    -  \nabla^2 \vE'
     &= - \ddO{t} \lp  
 \dd{\vE'}{t} + \vec{j}_\tx{eff}^\tx{sterile} - \mA^2 \vA' 
 \rp \\
     \partial_t^2 \vE' 
     -  \mA^2 \lp   \grad \phi' + \partial_t \vA' \rp
    -  \nabla^2 \vE'
     &= - \grad \rho_\tx{eff}^\tx{sterile} 
     - \partial_t \vec{j}_\tx{eff}^\tx{sterile}  \\
    \lp  \partial_t^2 +  \mA^2 -  \nabla^2 \rp \vE'
     &= - \grad \rho_\tx{eff}^\tx{sterile} 
     - \partial_t \vec{j}_\tx{eff}^\tx{sterile}  .
}
And as above, 
\eqa{
     \grad \rho_\tx{eff}^\tx{sterile} 
     + \partial_t \vec{j}_\tx{eff}^\tx{sterile}
     &= \grad \lb \e \grad \cdot \vE \rb 
+ \partial_t \lb - \e \dd{\vE}{t}
    + \e \grad \times \vB \rb \\
    &= \e \lb  \nabla^2 \vE 
     - \partial^2_t \vE\rb
}
so together we have 
\eqa{
\la{eq:double-wave-E}
    \lp  \partial^2_t - \nabla^2 \rp \vE &=  
    -\grad \rho_\tx{SM} - \partial_t \vec{j}_\tx{SM}
     +\e \lp  \partial^2_t - \nabla^2 \rp \vE' \\
    \lp  \partial_t^2 +  \mA^2 -  \nabla^2 \rp \vE'
     &= \e \lp  \partial^2_t  - \nabla^2 \rp \vE .
}
and we can further simplify 
\eqa{
    \lp  \partial^2_t - \nabla^2 \rp \vE &=  
    -\grad \rho_\tx{SM} - \partial_t \vec{j}_\tx{SM}
     +\e \lb - \mA^2 \vE' + \e \lp  \partial^2_t  - \nabla^2 \rp \vE \rb \\
   \lp  1 - \e^2 \rp 
   \lp  \partial^2_t - \nabla^2 \rp \vE &=  
    -\grad \rho_\tx{SM} - \partial_t \vec{j}_\tx{SM}
      - \e \mA^2 \vE' 
}
and our final pair of equation is 
\eqa{
\la{eq:active-E-wave}
   \lp  1 - \e^2 \rp 
   \lp  \partial^2_t - \nabla^2 \rp \vE &=  
    -\grad \rho_\tx{SM} - \partial_t \vec{j}_\tx{SM}
      - \e \mA^2 \vE' \\
\la{eq:sterile-E-wave}
    \lp  \partial_t^2 +  \mA^2 -  \nabla^2 \rp \vE'
     &= \e \lp  \partial^2_t  - \nabla^2 \rp \vE .
}
I would like to eliminate the $\partial_t^2 - \nabla^2$ operator on the right side of the sterile equation in favor of something simpler, while explicitly keeping the sources out of the sterile equation. 
We can so this at the cost of introducing a mass for the active mode. 
Define the field
\eqa{
    \vE_S = \vE' - \e E
}
then \Eq{eq:sterile-E-wave} becomes
\eqa{
    \lp  \partial_t^2 +  \mA^2 -  \nabla^2 \rp \vE'
     &= \e \lp  \partial^2_t  - \nabla^2 \rp \vE  \\
    \lp  \partial_t^2 +  \mA^2 -  \nabla^2 \rp \vE'
     &= \e \lp  \partial^2_t  - \nabla^2 + \mA^2 \rp \vE - \e \mA^2 \vE \\
    \lp  \partial_t^2 +  \mA^2 -  \nabla^2 \rp \lb \vE' - \e \vE \rb 
     &= - \e \mA^2 \vE \\
    \lp  \partial_t^2 +  \mA^2 -  \nabla^2 \rp \vE_S 
     &= - \e \mA^2 \vE \\
}
and \Eq{eq:active-E-wave} becomes
\eqa{
  \lp  1 - \e^2 \rp 
   \lp  \partial^2_t - \nabla^2 \rp \vE &=  
    -\grad \rho_\tx{SM} - \partial_t \vec{j}_\tx{SM}
      - \e \mA^2 \vE' \\
  \lp  1 - \e^2 \rp 
   \lp  \partial^2_t - \nabla^2 \rp \vE &=  
    -\grad \rho_\tx{SM} - \partial_t \vec{j}_\tx{SM}
      - \e \mA^2 \lp \vE_s + \e \vE \rp \\
  \lp  1 - \e^2 \rp 
   \lp  \partial^2_t - \nabla^2 \rp \vE + \e^2 \mA^2 \vE &=  
    -\grad \rho_\tx{SM} - \partial_t \vec{j}_\tx{SM}
      - \e \mA^2 \vE_S \\
}
Now rescale $\vE$ by 
\eqa{    
    \vE_A = \lp 1 - \e^2 \rp \vE
}
and we have the pair of equations 
\eqa{
  \lp  \partial^2_t + \frac{\e^2 \mA^2}{1 - \e^2} - \nabla^2 \rp \vE_A &=  
    -\grad \rho_\tx{SM} - \partial_t \vec{j}_\tx{SM}
      - \e \mA^2 \vE_S \\
    \lp  \partial_t^2 +  \mA^2 -  \nabla^2 \rp \vE_S 
     &= - \frac{\e \mA^2}{1 - \e^2} \vE_A.
}
And for symmetry lets rescale $\vE_S$ by 
\eqa{
    \vE_S' = \vE_S \sqrt{1 - \e^2}
}
so we have 
\eqa{
\la{eq:wave-massive-A}
  \lp  \partial^2_t + \frac{\e^2 \mA^2}{1 - \e^2} - \nabla^2 \rp \vE_A &=  
    -\grad \rho_\tx{SM} - \partial_t \vec{j}_\tx{SM}
      - \frac{\e \mA^2}{\sqrt{1 - \e^2}} \vE_S' \\
\la{eq:wave-massive-S'}
    \lp  \partial_t^2 +  \mA^2 -  \nabla^2 \rp \vE_S' 
     &= - \frac{\e \mA^2}{\sqrt{1 - \e^2}} \vE_A.
}
This is probably the prettiest set we find. 
How do these field relate to those of our original Lagrangian in \Eq{eq:interaction-L}?
\eqa{
    A_S' &= A_S \sqrt{1 - \e^2} 
    = \sqrt{1 - \e^2} \lp A' - \e A  \rp \\
    A_A &= \lp 1 - \e^2 \rp A   
}
This is like the mirror-image of the transformation in \Eqs{eq:tildeA}{eq:tildeA'}. 
There we shifted $A$ and only rescaled $A'$.  
By only scaling $A'$ and not shifting it we kept one massless mode. 
But here we have rescaled $A$ and shifted $A'$, which indeed introduces a mass for each mode and mass mixing. 
The mass mixing appears as the coupling in \Eqs{eq:wave-massive-A}{eq:wave-massive-S'}, which looks simpler than the kinetic mixing coupling of \Eqs{eq:active-E-wave}{eq:sterile-E-wave}.

As a check, we ought to be able to derive the decoupled, kinetic mode equations as well. 
We can immediately eliminate $\vE$ from \Eq{eq:sterile-E-wave} above, 
\eqa{
    \lp  \partial_t^2 +  \mA^2 -  \nabla^2 \rp \vE'
     &= \e \lp  \partial^2_t  - \nabla^2 \rp \vE 
     = \pfrac{\e}{1-\e^2} \lb -\grad \rho_\tx{SM} - \partial_t \vec{j}_\tx{SM}
      - \e \mA^2 \vE' \rb \\
    \lp  \partial_t^2 +  \frac{\mA^2}{1-\e^2} -  \nabla^2 \rp \vE'
     &= \pfrac{\e}{1-\e^2} 
     \lb -\grad \rho_\tx{SM} - \partial_t \vec{j}_\tx{SM} \rb \\
    \lp  \partial_t^2 +  \frac{\mA^2}{1-\e^2} -  \nabla^2 \rp \vtE'
     &= \frac{\e}{\sqrt{1-\e^2}} 
     \lb -\grad \rho_\tx{SM} - \partial_t \vec{j}_\tx{SM} \rb
}
and then return to \Eq{eq:double-wave-E}, which is 
\eqa{
    \lp  \partial^2_t - \nabla^2 \rp \lb \vE - \e \vE' \rb &=  
    -\grad \rho_\tx{SM} - \partial_t \vec{j}_\tx{SM}
}
so we have 
\eqa{
\la{eq:wave-kinetic-E}
    \lp  \partial^2_t - \nabla^2 \rp \vtE &=  
    -\grad \rho_\tx{SM} - \partial_t \vec{j}_\tx{SM} \\
\la{eq:wave-kinetic-E'}
    \lp  \partial_t^2 +  \frac{\mA^2}{1-\e^2} -  \nabla^2 \rp \vtE'
     &= \frac{\e}{\sqrt{1-\e^2}} 
     \lb -\grad \rho_\tx{SM} - \partial_t \vec{j}_\tx{SM} \rb
}
This is indeed what we would expect from \Eq{eq:kinetic-L}.

So we now have three pairs of electric field equations. 
Which is best?
If we know the SM source exactly, then \Eqs{eq:wave-kinetic-E}{eq:wave-kinetic-E'} are simplest to solve using the Coulomb and Yukawa Green's function, respectively. 
However, we may immediately know the SM field but not the sources directly, e.g.~for a driven cavity mode. 
We could find the sources from the SM field and then use \Eqs{eq:wave-kinetic-E}{eq:wave-kinetic-E'}. 
Or we can work perturbativley in $\e$ from \Eqs{eq:active-E-wave}{eq:sterile-E-wave}. 
The known $\vE$ and unknown SM sources solve \Eq{eq:active-E-wave} to leading order, $\Order(\e^0)$. 
Then we find $\vE'$ at $\Order\lp \e\rp$ using $\lp  \partial_t^2 - \nabla^2\rp \vE$ as its source.
Finally we find the $\Order\lp \e^2\rp$ correction to our original $\vE$ by taking $\e \mA^2 \vE'$ as the source for the correction. 
One could apply the same procedure to \Eqs{eq:wave-massive-A}{eq:wave-massive-S'}, at leading order ignoring the mass term for $\vE_A$ the adopting massless cavity modes. 
Then for the $\Order(\e^2)$ correction, include both the $\vE_S'$ and the mass term as the effective sources for the correction. 
I believe this is the approach of \cite{Graham:2014sha} (though they may just ignore the mass correction completely).

\section{Cavity LSW}

\subsection{Cavity Modes}

Within an empty cavity the electric field satisfies 
\eqa{
    \lp \partial_t^2 - \nabla^2 \rp \vE = 0. 
}
If $\vE_p$ is a mode of frequency $\omega_p$, then  $\vE_P$ is an eigenvector of the $\nabla^2$ operator, 
\eqa{
\la{eq:eigenvector}
    \nabla^2 \vE_p = -\omega_p^2 \vE_p
}
and choose the normalization to be 
\eqa{
    \int_\tx{cavity} \diffp{3}{x} \lvl \vE_p \rvl^2 = V
} 
where V is the cavity volume. 
Then $\vE_p$ is unitless, $\mathcal{E}_p$ has units of electric field and is a characteristic field value to which the mode $p$ is driven. 
To be more specific, we can relate this to the time-averaged energy storage in the mode $p$. 
The energy density in the EM field is 
\eqa{
    u = \frac{1}{2} \lp |E|^2 + |B|^2 \rp
}
and for a cavity mode the time-average energy in the electric field equals that of the magnetic field, and the total energy is constant in time. 
So the stored energy is equal to that of the peak energy in the electric field, and thus we can just ignore the time dependence. 
\eqa{
\la{eq:stored-energy}
    U_p = \frac{1}{2} \int_\tx{cavity} \diffp{3}{x} |\vE_p|^2 
      &= \frac{1}{2} \mathcal{E}_p^2  V
}

Now suppose there is a source $\vec{s} = -\grad \rho - \partial_t \vec{j}$ that oscillates at a fixed frequency $\omega$ within the cavity. 
Then the driven field satisfies
\eqa{
    \lp \partial_t^2 - \nabla^2 \rp \vE = \vec{s} e^{i \omega t} .
}
Suppose the modes $\vE_p$ form a complete, orthogonal set within the cavity. 
It follows that we can expand 
\eqa{
    \vE = \Sigma_p \mathcal{E}_p \vE_p e^{i \omega_p t}.
}
But orthogonality, the energy is partitioned distinctly into modes 
\eqa{
    U =  \frac{1}{2} V \,  \Sigma_p \mathcal{E}_p^2  .
}
The excitation of each mode is 
\eqa{
    \lp \partial_t^2 - \nabla^2 \rp 
    \Sigma_p \mathcal{E}_p \vE_p e^{i \omega t} 
    &= \vec{s} e^{i \omega t} \\
   \Sigma_p \mathcal{E}_p 
   \lp -\omega^2 + \omega_p^2  \rp \vE_p e^{i \omega t} 
    &= \vec{s} e^{i \omega t} \\
   \mathcal{E}_p  \lp -\omega^2 + \omega_p^2  \rp  V
    &= \int_\tx{cavity} \diffp{3}{x}  \lb \vec{s} \cdot  \vE_p \rb 
}
finally
\eqa{
   \mathcal{E}_p &= 
   \frac{1}{ \omega_p^2 -\omega^2 } \,
   \lp \frac{1}{V} \int_\tx{cavity} \diffp{3}{x} \vec{s} \cdot \vE_p \rp
}

Now let's include dissipation.  
We can artificially add a term to the wave equation, pretending that the interior of the cavity is filled with an ohmic conductor of conductivity $\sigma$. 
This generates a dissipation current $\vec{j}_d = \sigma \vE$ and the field obey,  
\eqa{
    \lp \partial_t^2 - \nabla^2 \rp \vE &= - \partial_t \vec{j}_d \\ 
    \lp \partial_t^2 - \nabla^2 \rp \vE &= - \sigma \partial_t \vE \\ 
    \lp \partial_t^2 + \sigma \partial_t - \nabla^2 \rp \vE &= 0.
}
Now the energy dissipated from the $p$ mode is 
\eqa{
    P &= \int_\tx{cavity} \diffp{3}{x} \vE \cdot \vec{j}_d
    = \sigma \int_\tx{cavity} \diffp{3}{x} \lvl \vE \rvl^2 
    = \sigma \mathcal{E}_p^2 V \cos^2\lp \omega_p t \rp \\
  \int_\tx{cycle} \diff{t} P &= \sigma \mathcal{E}_p^2 V \frac{\pi}{\omega_p}
  = \sigma U_p \frac{2\pi}{\omega_p}
}
and the $Q$ factor is defined to be 
\eqa{
    Q = \frac{2 \pi U }{\int_\tx{cycle} \diff{t} P} 
    = \frac{\omega U}{\bar{P}}
}
where $\bar{P}$ is the time-averaged dissipation rate
\eqa{
    \bar{P} = 
    \frac{\omega}{2 \pi} \int_\tx{cycle} \diff{t} P
    = \sigma \mathcal{E}_p^2 V \frac{1}{2}
    = \sigma U_p.
}
The $Q$ factor related to $\sigma$ as 
\eqa{
\la{eq:sigma-to-Q}
    Q = \frac{\omega_p}{\sigma}.
}
and the field equation can be written
\eqa{
    \lp \partial_t^2 
    + \frac{\omega_p}{Q} \partial_t - \nabla^2 \rp \vE &= 0 .
}
If the dissipation was truly dominated by a space-filling ohmic conductor, then this equation would hold for all modes with $Q$ varying linearly in frequency as \Eq{eq:sigma-to-Q}.  
In reality the dissipation is more complicated, due I think to the fact that the modes field penetrate some skin depth into the wall which have some small resistivity. 
But we can treat $Q$ as some phenomenological parameter that varies with mode in a complicated way. 

We now expand the field generated by a source $\vec{s}$ in the same modes $\vE_p$ as above, obeying \Eq{eq:eigenvector}. 
\eqa{
    \lp \partial_t^2 
    + \frac{\omega_p}{Q} \partial_t - \nabla^2 \rp 
    \Sigma_p \mathcal{E}_p \vE_p &= \vec{s} \\
    \Sigma_p \mathcal{E}_p \lp -\omega^2
    + i \frac{\omega_p}{Q} \omega + \omega_p^2 \rp 
     \vE_p &= \vec{s} \\
    V \mathcal{E}_p \lp -\omega^2
    + i \frac{\omega_p}{Q} \omega + \omega_p^2 \rp 
     &= \int_\tx{cavity} \diffp{3}{x}  \vec{s} \cdot \vE_p .
}
Note the bandwidth is as expected. 
Expand $\omega = \omega_p + \delta \omega$, 
\eqa{
    V \mathcal{E}_p \lb -\lp \omega_p^2 + 2 \delta \omega \omega_p\rp
    + i \frac{\omega_p}{Q} \lp \omega_p +\delta \omega\rp + \omega_p^2 \rb
     &\approx \int_\tx{cavity} \diffp{3}{x}  \vec{s} \cdot \vE_p \\
    V \mathcal{E}_p \omega_p \lb -2 \delta \omega
    + i \frac{1}{Q} \lp \omega_p + \delta \omega\rp  \rb
     &\approx \int_\tx{cavity} \diffp{3}{x}  \vec{s} \cdot \vE_p \\
    V \mathcal{E}_p \omega_p \lb -2 \delta \omega
    + i \frac{\omega_p}{Q}  \rb
     &\approx \int_\tx{cavity} \diffp{3}{x}  \vec{s} \cdot \vE_p \\
    V^2 \mathcal{E}_p^2 \omega_p^2 
    \lb 4 \delta \omega^2
    + \frac{\omega_p^2}{Q^2}  \rb
     &\approx \lvl \int_\tx{cavity} \diffp{3}{x}  \vec{s} \cdot \vE_p \rvl^2.
}
The lineshape is 
\eqa{
    \mathcal{E}_p^2     
     &\approx 
     \pfrac{Q^2}{V^2 \omega_p^4}
     \pfrac{1}
     {\frac{4 Q^2 \delta \omega^2}{\omega_p^2} + 1}
     \lvl \int_\tx{cavity} \diffp{3}{x}  \vec{s} \cdot \vE_p \rvl^2.
}
and we have resonance when $\delta \omega \ll \omega_p/2Q$, which yields
\eqa{
\la{eq:res-Q-amplitude}
    \mathcal{E}_p = 
    \frac{- i Q}{\omega_p^2}
    \lp \frac{1}{V} \int_\tx{cavity} \diffp{3}{x} 
    \vec{s} \cdot \vE_p \rp.
}

The energy stored on resonance is 
\eqa{
    U_p = 
    \frac{Q^2}{2 \omega_p^4 V}
    \lvl \int_\tx{cavity} \diffp{3}{x} 
    \vec{s} \cdot \vE_p \rvl^2
}
and the available power is 
\eqa{
    \bar{P} =  \frac{Q}{2 \omega_p^3 V}
    \lvl \int_\tx{cavity} \diffp{3}{x} 
    \vec{s} \cdot \vE_p \rvl^2.
}

\subsection{Hidden Photon Mixing}

Consider two EM cavities, one of which is pumped with power and other is not. 
The unpumped cavity will be rung up due to the hidden photon. 
This is easily calculated with the mass-mixing basis, \Eqs{eq:wave-massive-A}{eq:wave-massive-S'} repeated here
\eqa{
  \lp  \partial^2_t + \frac{\e^2 \mA^2}{1 - \e^2} - \nabla^2 \rp \vE_A &=  
    -\grad \rho_\tx{SM} - \partial_t \vec{j}_\tx{SM}
      - \frac{\e \mA^2}{\sqrt{1 - \e^2}} \vE_S' \\
\la{eq:wave-massive-S'-3}
    \lp  \partial_t^2 +  \mA^2 -  \nabla^2 \rp \vE_S' 
     &= - \frac{\e \mA^2}{\sqrt{1 - \e^2}} \vE_A.
}
To zeroth order in $\e$ we have $\vE_S'=0$ and $\vE_A = \vE_0$ is some known, massless, excited cavity mode (or sum of modes). 
This satisfies 
\eqa{
    \lp  \partial^2_t - \nabla^2 \rp \vE_0 &=  
    -\grad \rho_0 - \partial_t \vec{j}_0
}
where $\rho_0$, $\vec{j}_0$ are the sources within the cavity and on its walls that drive $\vE_0$ and maintain its boundary conditions. 

$\vE_0$ sources the $\vE_S'$ field. 
Assume that $\vE_0$ oscillates in time as $e^{i \omega t}$, then to order $\e^2$ we have 
\eqa{
    \lp  \partial_t^2 +  \mA^2 -  \nabla^2 \rp \vE_S' 
     &\approx - \e \mA^2 \vE_0 e^{i \omega t} \\
\la{eq:sourced-source}
\Rightarrow \qq
    \vE_S' &= \frac{- \e \mA^2 }{4 \pi} \int \diffp{3}{x_s}
    \frac{\vE_0(\vec{x}_s)}{\lvl \vec{x}_s - \vec{r}\rvl}
    e^{- k \lvl \vec{x}_s - \vec{r}\rvl + i \omega t}
}
where $k = \sqrt{\mA^2-\omega^2}$ and the integral over $\vec{x}_s$ is taken over the volume of the driven, source cavity. 
Note when $0 \leq \omega < \mA$ we have $0 < k \leq \mA$ and $\vE_S'$ is exponentially decaying. 
If $\mA < \omega$ then $k=iq$, where $q>0$ so that $\vE_S'$ is an outward-going spherical wave at large distances from the source cavity.

$\vE_S'$ is not shielded from the detection cavity. 
This is because the Lorentz force law in this basis is 
\eqa{
    \vec{F} = q \vE_A + q \vec{v} \times \vB_A
}.
i.e., $\vE_S$ drives no screening currents and thus takes no notice of conductors. 
Within the detection cavity, \Eq{eq:sourced-source} is a source for an additional $\vE_A$ field. 
This is field is generated only by sources within the cavity, i.e. $\vE_S$ within the cavity, as the walls will screen anything external sources. 
Inside the cavity we have 
\eqa{
\lp  \partial^2_t + \frac{\e^2 \mA^2}{1 - \e^2} - \nabla^2 \rp \vE_A &=  
      - \frac{\e \mA^2}{\sqrt{1 - \e^2}} \vE_S' \\
\lp  \partial^2_t  - \nabla^2 \rp \vE_A &=  
      - \frac{\e \mA^2}{\sqrt{1 - \e^2}} \vE_S'
      - \frac{\e^2 \mA^2}{1 - \e^2}\vE_A
}
where we treat the mass term pertrubatively as an effective source. 
We can in fact ignore it - the sterile source term is order $\sim \e E_s' \sim \e^2$, so at leading order $E_A \sim \e^2$ and the mass-term source is $\sim \e^2 E_A^2 \sim \e^4$. 
So we have to leading order in $\e$ 
\eqa{
\lp  \partial^2_t  - \nabla^2 \rp \vE_A &\approx  
      - \e \mA^2 \vE_S'
}
where $\vE_S'$ is given by \Eq{eq:sourced-source}.
The excitation of a cavity mode by this source is given by \Eq{eq:res-Q-amplitude} where $\vec{s} = - \e \mA^2 \vE_S'$:
\eqa{
    \mathcal{E}_p = 
    \frac{i Q \e \mA^2}{\omega_p^2}
    \lp \frac{1}{V} \int \diffp{3}{x_d} 
    \vE_S' \cdot \vE_p \rp.
}
and the integral is over the volume of the detection cavity. 
Now insert \Eq{eq:sourced-source}, 
\eqa{
    \mathcal{E}_p &= 
    \frac{i Q \e \mA^2}{\omega_p^2}
    \lp \frac{1}{V} \int \diffp{3}{x_d} 
    \vE_p(\vec{x}_d) \cdot
    \lb \frac{- \e \mA^2 }{4 \pi} \int \diffp{3}{x_s}
    \frac{\vE_0(\vec{x}_s)}{\lvl \vec{x}_s - \vec{x}_d \rvl}
    e^{- k \lvl \vec{x}_s - \vec{x}_d \rvl} \rb \rp \\
   &=  \frac{-i Q \e^2 \mA^4}{4 \pi \omega_p^2 V}
    \lp \int \diffp{3}{x_d} \int \diffp{3}{x_s} 
    \vE_p(\vec{x}_d) \cdot
    \frac{\vE_0(\vec{x}_s)}{\lvl \vec{x}_s - \vec{x}_d \rvl}
    e^{- k \lvl \vec{x}_s - \vec{x}_d \rvl} \rp.
}
Lets relabel so that all source cavity quantities are denoted $s$ and detection cavity $d$, 
\eqa{
    \mathcal{E}_d 
   &=  \frac{-i Q_d \e^2 \mA^4}{4 \pi \omega^2 V_d} 
   \mathcal{E}_s
    \lp \int \diffp{3}{x_d} \int \diffp{3}{x_s} 
    \frac{\vE_d(\vec{x}_d) \cdot \vE_s(\vec{x}_s)}
    {\lvl \vec{x}_s - \vec{x}_d \rvl}
    e^{- k \lvl \vec{x}_s - \vec{x}_d \rvl} \rp.
}
This has the correct units. 
Recall that the mode functions $\vE_s$ and $\vE_d$ are unitless. 
The available power in the detection cavity is $P = \omega U /Q = \mathcal{E}^2 V \omega/ 2 Q$.
\eqa{
\la{eq:general-signal-power}
    \bar{P}_d 
   &=  \frac{Q_d \e^4 \mA^8}{32 \pi^2 \omega^3 V_d} 
   \mathcal{E}_s^2 
   \lvl \int \diffp{3}{x_d} \int \diffp{3}{x_s} 
    \frac{\vE_d(\vec{x}_d) \cdot \vE_s(\vec{x}_s)}
    {\lvl \vec{x}_s - \vec{x}_d \rvl}
    e^{- \sqrt{m^2 - \omega^2} \lvl \vec{x}_s - \vec{x}_d \rvl} \rvl^2.
}

\subsection{Partitioned Box}

Consider a rectangular cavity which is partitioned by a conducting wall. 
Let the $z$ direction be perpendicular to the partition, and let the two cavities have square cross-section $a \times a$ in the $x$ and $y$ directions and length $H$ in the $z$ direction. 
Let the wall have thickness $\tau$, so that the total dimension of the box are $a \times a \times \lp 2 H + \tau \rp$, ignoring the thickness of the outer walls. 

\subsubsection{Scaling Estimate}
\la{eq:part-box-estimate}

Lets do the integral in \Eq{eq:general-signal-power}.
Consider specifically the limit $\omega \ll m$ so that $k \approx m$ and further that $m^{1} \ll a, H$.
That is, the hidden photon is very short range compared to the sizes of the cavities.  
In this limit we have the following estimate. 
The integral has support only when $|\vec{x}_s - \vec{x}_d| < m^{-1}$. 
At fixed $\vec{x}_d$, we have a contribution only from those $\vec{x}_s$ within the volume $m^{-3}$ around $\vec{x}_d$.  
There are no such $\vec{x}_s$ for most points $\vec{x}_d$ in the detection cavity since $\vec{x}_s$ is in the source cavity.  
If $\tau m \gg 1$ then there are no such points, but if $\tau m \ll 1$ then there is thin sheet of volume $a^2 m^{-1}$ adjacent to the partition where the integral has support. 
Within that sheet we have at fixed $\vec{x}_d$
\eqa{
    \int \diffp{3}{x_s} 
    \frac{\vE_s(\vec{x}_s)}
    {\lvl \vec{x}_s - \vec{x}_d \rvl}
    e^{- k \lvl \vec{x}_s - \vec{x}_d \rvl} 
    &\sim 
    \int_{\tx{hemisphere } m^{-1} } \diffp{3}{x_s} 
   \frac{ \vE_s(x_s, y_s, a)}{m^{-1}} \\
   &\sim \frac{2 \pi}{3} m^{-2} E_{s, \tx{wall}}
} 
where $ E_{s,\tx{wall}}$ denotes the typical value of the source mode function on the partition-wall of the cavity. 
Now integrate over the detection mode, 
\eqa{
    \int \diffp{3}{x_d} \vE_d(\vec{x}_d) 
    \lb \int \diffp{3}{x_s} 
    \frac{\vE_s(\vec{x}_s)}
    {\lvl \vec{x}_s - \vec{x}_d \rvl}
    e^{- k \lvl \vec{x}_s - \vec{x}_d \rvl} \rb
    &\sim \int_{\tx{sheet } a^2 m^{-1}} \diffp{3}{x_d} \vE_d(\vec{x}_d) 
      \lb \frac{2 \pi}{3} m^{-2} E_{s, \tx{wall}} \rb \\
    &\sim \frac{2 \pi}{3} a^2 m^{-3} E_{s, \tx{wall}} E_{d, \tx{wall}}
} 
The mode function are unitless, and $\sim \Order(1)$ as we have normalized them to the total volume. 
With foresight we take $E_{d, \tx{wall}} \cdot E_{s, \tx{wall}} \approx 2$ (see \Eq{eq:optimal-area-overlap}). 
Then
\eqa{
\int \diffp{3}{x_d} \int \diffp{3}{x_s} 
    \frac{\vE_d(\vec{x}_d) \cdot \vE_s(\vec{x}_s)}
    {\lvl \vec{x}_s - \vec{x}_d \rvl}
    e^{- k \lvl \vec{x}_s - \vec{x}_d \rvl} 
    \approx \frac{4 \pi}{3} \frac{a^2}{m^3}
}
The signal power is 
\eqa{
\la{eq:Ps-estimate-highmass}
    P_d \approx \frac{Q_d \e^4 \mA^8}{32 \pi^2 \omega^3 a^2 H} 
   \mathcal{E}_s^2 \cdot \frac{16 \pi^2}{9} \frac{a^4}{m^6} 
   \approx \frac{Q_d \e^4 \mA^2 a^2 \mathcal{E}_s^2}
   {18 \, \omega^3 H} 
   \qq \qq \lb m \gg \omega \rb
}
In the limit that $m \ll \omega, a^{1} H^{-1}$, the integral out to evaluate to 
\eqa{
\int \diffp{3}{x_d} \int \diffp{3}{x_s} 
    \frac{\vE_d(\vec{x}_d) \cdot \vE_s(\vec{x}_s)}
    {\lvl \vec{x}_s - \vec{x}_d \rvl}
    e^{- k \lvl \vec{x}_s - \vec{x}_d \rvl} 
    \sim \frac{a^2 H \cdot a^2 H}{H} E_{s, \tx{interior}} E_{d, \tx{interior}}
    \sim a^4 H
}
where $E_{s, \tx{interior}}$ and $E_{d, \tx{interior}}$ are now the typical values of the mode functions over the interior and I've taken their product to be $\approx 1$ as the modes are normalized to have a volume-averaged value of $1$.
The power is 
\eqa{
\la{eq:Ps-estimate-lowmass}
    P_d \approx \frac{Q_d \e^4 \mA^8}{32 \pi^2 \omega^3 a^2 H} 
   \mathcal{E}_s^2 \cdot a^8 H^2
   \approx \frac{Q_d \e^4 \mA^8 a^6 H \mathcal{E}_s^2}
   {32 \pi^2 \, \omega^3} 
   \qq \qq \lb m \ll \omega \rb
}


\subsubsection{Rectangular Cavity Modes}

Let the modes have time dependence 
\eqa{
    \vec{E}(\vec{r}, t) &= \vec{E}_p(\vec{r}) \,  e^{i \omega_p t} \\
    \vec{B}(\vec{r}, t) &= \vec{B}_p(\vec{r}) \,  e^{i \omega_p t}.
}
Then we have the following TM modes
\eqa{
\la{eq:E_TM_x}
    E_x &= \frac{ -\sqrt{8} \, k_z k_x}{\omega \sqrt{k_x^2 + k_y^2}}
    \cos\lp k_x x \rp  \sin\lp k_y y \rp  \sin\lp k_z z \rp \\
\la{eq:E_TM_y}
    E_y &= \frac{ -\sqrt{8} \, k_z k_y}{\omega \sqrt{k_x^2 + k_y^2}}
    \sin\lp k_x x \rp  \cos\lp k_y y \rp  \sin\lp k_z z \rp \\
\la{eq:E_TM_z}
    E_z &= \frac{ \sqrt{8} \, \sqrt{k_x^2 + k_y^2}}{\omega}
    \sin\lp k_x x \rp  \sin\lp k_y y \rp  \cos\lp k_z z \rp \\
}
and TE 
\eqa{
\la{eq:E_TE_x}
    E_x &= \frac{ i \sqrt{8} \, k_y}{\sqrt{k_x^2 + k_y^2}}
    \cos\lp k_x x \rp  \sin\lp k_y y \rp  \sin\lp k_z z \rp \\
\la{eq:E_TE_y}
    E_y &= \frac{-i \sqrt{8} \, k_x}{\sqrt{k_x^2 + k_y^2}}
    \sin\lp k_x x \rp  \cos\lp k_y y \rp  \sin\lp k_z z \rp \\
\la{eq:E_TE_z}
    E_z &= 0
}
with the frequency 
\eqa{
    \omega = \sqrt{ k_x^2 + k_y^2 + k_z^2 }.
}
The magnetic fields are given by Farday's law, 
\eqa{
    \grad \times \vE &= -\partial_t \vB = - i \omega \vB \\
    \vB &= \frac{i}{\omega} \grad \times \vE.
}
The resonance condition is that a half-integer number of wavelength fit inside the box in all directions, 
\eqa{
    k_x &= \frac{\pi}{a} n_x \\
    k_y &= \frac{\pi}{a} n_y \\
    k_z &= \frac{\pi}{H} n_z
}
where at most two of the $n_i$ may vanish. 
The above modes are normalized to the volume
\eqa{
\la{eq:norm}
    \int \diffp{3}{x} |\vec{E}|^2
    = \int \diffp{3}{x} |\vec{B}|^2 = a^2 H.
}
See mathematica "RectangularCavity.nb" for verification that the above are indeed solutions of Maxwell's equations with the correct boundary conditions and that they satisfy \Eq{eq:norm}.

\subsubsection{Optimal Mode Choice}

Lets choose a pair of source and detection modes optimized in the limit $m \gg \omega, a^{-1}, H^{-1}$. 
From the estimate in \Sec{eq:part-box-estimate}, we see that the signal will scale in this limit as the square of the area integral 
\eqa{
    I_\tx{wall} = \frac{1}{a^2} 
    \int_\tx{wall} \diff{x}\diff{y} \vE_s \cdot \vE_d
}
that is, the average of the overlap of the electric fields on the partition wall. 

We want to maximize the magnitude of the electric field along the partitioned wall, which is $z=0$ for the detection cavity and $z=H/a$ for the source cavity.
This vanishes for TE as this is exactly the conducting wall's boundary condition --- the electric field must be perpendicular to the wall. 
So we need TM. 
For those modes (for the same reason) the only nonzero component of electric field will be in the $z$ direction. 
Looking at the mode function in Eqs.~\ref{eq:E_TM_x}-\ref{eq:E_TE_z}, this is clear as $\sin\lp \pi n_z \frac{z}{H} \rp$ vanishes when $z=0$ or $z=H$. 
Further, $\cos\lp \pi n_z \frac{z}{H} \rp$ is $\pm 1$ when $z=0$ or $z=H$, so we have that
\eqa{
    \lb \vE_s \cdot \vE_d \rb_\tx{wall} &= 
    \lb \frac{ \sqrt{8} \, \sqrt{k_x^2 + k_y^2}}{\omega}
    \sin\lp \pi n_x \frac{x}{a}\rp  \sin\lp \pi n_y \frac{y}{a} \rp 
    \cos\lp \pi n_z \rp \rb
    \lb \frac{ \sqrt{8} \, \sqrt{k_x^{'2} + k_y^{'2}}}{\omega'}
    \sin\lp \pi n'_x \frac{x}{a} \rp  \sin\lp \pi n'_y  \frac{y}{a}\rp \rb \\
    &=
    \lp -1 \rp^{n_z}
    \frac{ 8 \, \sqrt{k_x^2 + k_y^2} \sqrt{k_x^{'2} + k_y^{'2}}}
    {\omega \omega'}
    \sin\lp \pi n_x \frac{x}{a} \rp  \sin\lp \pi n'_x \frac{x}{a} \rp 
    \sin\lp \pi n_y \frac{y}{a} \rp \sin\lp \pi n'_y \frac{y}{a} \rp .
}
Here primes indicate the detection cavity and unprimed the source cavity. 
Now $\sin$ is orthogonal on this interval, so we must choose $n_x = n'_x$ and $n_y = n'_y$. 
We also cannot chose either of them to vanish. 
It is a fact for non-zero integers $n$ and $m$ that
\eqa{
    \int_0^a \diff{x} \sin\lp n \frac{x}{a} \rp \sin\lp m \frac{x}{a} \rp 
    = \frac{a}{2} \delta_{n,m}
}
so we have 
\eqa{
    \int_\tx{wall} \diff{x}\diff{y} \vE_s \cdot \vE_d 
    = \lp -1 \rp^{n_z} \, 2a^2 \, \frac{\lp k_x^2 + k_y^2 \rp}{\omega^2}
    &= \lp -1 \rp^{n_z} \, 2a^2 \, 
    \lp 1 - \frac{k_z^2}{\omega^2} \rp \\
\la{eq:optimal-area-overlap}
    \Rightarrow \qq
    \lvl I_\tx{wall} \rvl &\leq 2
}
where the inequality is saturated if we choose $n_z = 0$. 
Our best signal thus comes from using identical modes in each cavity, and using any of the $\tx{TM}_{n, m, 0}$ modes. 
It looks like it should not matter which $n$ and $m$ we choose, so for simplicity we can take $\tx{TM}_{110}$. 

\subsubsection{Numerical Overlap Integrals}

Now as a check on \Eqs{eq:Ps-estimate-highmass}{eq:Ps-estimate-lowmass} we can do the overlap integral numerically. 
This will also give us the exact extrapolation in the limit $m \approx \omega$, which we need make a pretty estimate of the reach. 

Let's focus the formalism on the regime $m\gg\omega$ and normalize the integral as 
\eqa{
    G = \frac{m^3}{a^2} \int \diffp{3}{x_d} \int \diffp{3}{x_s} 
    \frac{\vE_d(\vec{x}_d) \cdot \vE_s(\vec{x}_s)}
    {\lvl \vec{x}_s - \vec{x}_d \rvl}
    e^{- k \lvl \vec{x}_s - \vec{x}_d \rvl} 
}
so the signal power is 
\eqa{
\la{eq:Psig}
    \bar{P}_d 
   &=  \frac{Q_d \e^4 \mA^2 a^2 \mathcal{E}_s^2 }{32 \pi^2 \omega^3 H} 
   \lvl G \rvl^2.
}
and we expect $G$ is a constant $G \sim 0.5$ for a good choice of modes when $m\gg\omega$.
Let write the integral in term of coordinates centered on the lower left corner of each sub-cavity, so the integrals both run over $0 < x < a$, $0<y<a$, and $0<z<H$. 
Let $\vec{x}_s = \vec{r}_s$ have origin on the corner of the source sub-cavity. 
Then let $\vec{x}_d = (H + \tau) \hat{z} + \vec{r}_d$, where $\vec{r}_d$ has origin on the corner of the detection sub-cavity. 
Then 
\eqa{
    \lvl \vec{x}_s  - \vec{x}_d \rvl^2 = 
    \lp x_d - x_s \rp^2
    + \lp y_d - y_s \rp^2
    + \lp H + \tau + z_d - z_s \rp^2 
    \equiv \Delta^2
}
and 
\eqa{
    G = \frac{m^3}{a^2} 
    \int_0^a \diff{x_d} \int_0^a \diff{y_d} \int_0^H \diff{z_d} 
    \int_0^a \diff{x_s} \int_0^a \diff{y_s} \int_0^H \diff{z_s} 
    \vE_d(\vec{r}_d) \cdot \vE_s(\vec{r}_s)
    \frac{e^{- k \Delta }}{\Delta}
}
Scale the coordinates $x$ and $y$ by $a$ and $z$ by $H$ so that they are all unitless and run over $[0,1]$, 
\eqa{
    G &= m^3 a H^2 
    \int_0^1 \diff{x_d} \int_0^1 \diff{y_d} \int_0^1 \diff{z_d} 
    \int_0^1 \diff{x_s} \int_0^1 \diff{y_s} \int_0^1 \diff{z_s} 
    \vE_d(\vec{r}_d) \cdot \vE_s(\vec{r}_s)
    \frac{e^{- k a \delta }}{\delta} \\
    \delta &= \sqrt{\lp x_d - x_s \rp^2
    + \lp y_d - y_s \rp^2
    + \lp \frac{H}{a} \lb 1 + z_d - z_s \rb + \frac{\tau}{a} \rp^2 }.
}
Now let's pick some cavity modes. 
In these units the modes are for TM
\eqa{
    E^\tx{TM}_x &= \frac{ -\sqrt{8} \, k_z k_x}{\omega \sqrt{k_x^2 + k_y^2}}
    \cos\lp \pi n_x x \rp  \sin\lp \pi n_y y \rp 
    \sin\lp \pi n_z z \rp \\
    E^\tx{TM}_y &= \frac{ -\sqrt{8} \, k_z k_y}{\omega \sqrt{k_x^2 + k_y^2}}
    \sin\lp \pi n_x x \rp  \cos\lp \pi n_y y \rp 
    \sin\lp \pi n_z z \rp \\
    E^\tx{TM}_z &= \frac{ \sqrt{8} \, \sqrt{k_x^2 + k_y^2}}{\omega}
    \sin\lp \pi n_x x \rp  \sin\lp \pi n_y y \rp 
    \cos\lp \pi n_z z \rp
}
and TE 
\eqa{
    E^\tx{TE}_x &= \frac{ i \sqrt{8} \, k_y}{\sqrt{k_x^2 + k_y^2}}
    \cos\lp \pi n_x x \rp  \sin\lp \pi n_y y \rp 
    \sin\lp \pi n_z z \rp \\
    E^\tx{TE}_y &= \frac{-i \sqrt{8} \, k_x}{\sqrt{k_x^2 + k_y^2}}
    \sin\lp \pi n_x x \rp  \cos\lp \pi n_y y \rp 
    \sin\lp \pi n_z z \rp \\
    E^\tx{TE}_z &= 0. 
}
We know that we want $\tx{TM}_{110}$, but for thoroughness we can just as easily evaluate all the overlaps of the first few modes. 
Lets write the prefactors entirely in terms of $n_i$, and $H/a$. 
E.g., 
\eqa{
    \frac{ -\sqrt{8} \, k_z k_x}{\omega \sqrt{k_x^2 + k_y^2}} 
    &= -\sqrt{8} \pfrac{\pi n_z}{H} \pfrac{\pi n_x}{a}
    \frac{1}{
    \sqrt{ \pfrac{\pi n_x}{a}^2 +  \pfrac{\pi n_y}{a}^2+\pfrac{\pi n_z}{H}^2}
    \sqrt{ \pfrac{\pi n_x}{a}^2 +  \pfrac{\pi n_y}{a}^2}} \\
    &= -\sqrt{8} \pfrac{n_z a}{H} n_x
    \frac{1}
    {\sqrt{ n_x^2 +  n_y^2 + \pfrac{a n_z}{H}^2} \sqrt{ n_x^2 +  n_y^2}}\\
    &= \frac{-\sqrt{8} \, n_z n_x}
    {\sqrt{ \pfrac{H}{a}^2 \lp n_x^2 +  n_y^2 \rp +n_z^2} 
    \sqrt{ n_x^2 +  n_y^2}}.
}
The modes are 
\eqa{
    E^\tx{TM}_x &= 
    \frac{-\sqrt{8} \, n_z n_x}
    {\sqrt{ \pfrac{H}{a}^2 \lp n_x^2 +  n_y^2 \rp +n_z^2} 
    \sqrt{ n_x^2 +  n_y^2}}
    \cos\lp \pi n_x x \rp  \sin\lp \pi n_y y \rp 
    \sin\lp \pi n_z z \rp \\
    E^\tx{TM}_y &= -\frac{-\sqrt{8} \, n_z n_y}
    {\sqrt{ \pfrac{H}{a}^2 \lp n_x^2 +  n_y^2 \rp +n_z^2} 
    \sqrt{ n_x^2 +  n_y^2}}
    \sin\lp \pi n_x x \rp  \cos\lp \pi n_y y \rp 
    \sin\lp \pi n_z z \rp \\
    E^\tx{TM}_z &= \sqrt{8}\, \pfrac{H}{a} \, 
    \frac{ \sqrt{n_x^2 + n_y^2}}
    {\sqrt{ \pfrac{H}{a}^2 \lp n_x^2 +  n_y^2 \rp +n_z^2}}
    \sin\lp \pi n_x x \rp  \sin\lp \pi n_y y \rp 
    \cos\lp \pi n_z z \rp
}
and 
\eqa{
    E^\tx{TE}_x &=  i \sqrt{8} \, \frac{n_y}{\sqrt{n_x^2 + n_y^2}}
    \cos\lp \pi n_x x \rp  \sin\lp \pi n_y y \rp 
    \sin\lp \pi n_z z \rp \\
    E^\tx{TE}_y &= -i \sqrt{8} \, \frac{n_x}{\sqrt{n_x^2 + n_y^2}}
    \sin\lp \pi n_x x \rp  \cos\lp \pi n_y y \rp 
    \sin\lp \pi n_z z \rp \\
    E^\tx{TE}_z &= 0. 
}
The normalization in these units is 
\eqa{
    \int \diffp{3}{x} \lvl E(\vec{x}) \rvl &= a^2 H \\
    \Rightarrow \qq 
    \int \diffp{3}{x'} \lvl E(\vec{x}') \rvl &= 1 
}
where $x'_i = x_i/L_i$. 
{\color{red} Note: this normalization is off, there should be some extra factors of $\sqrt{2}$ when some of the $n_i=0$.
The above was inadvertently derived assuming all $n_i \neq 0$.
Having some $n_i$ vanish is a special case because of the integral
\eqa{
    \int_0^1 \diff{x} \sin^2 \lp \pi n x \rp = 
    \begin{cases}
      1 & n = 0 \\
      \frac{1}{2} & n > 0 
    \end{cases}
}}

\subsubsection{$\tx{TM}_{110}$}

Let's evaluate the signal power fully for the $\tx{TM}_{110}$ modes. 
In principle we can drive the source cavity up to the critical field of the material, $B_c \approx 0.1 \Tesla$. 
We ought to write \Eq{eq:Psig} in term of the magnetic field at the surface of the cavity, not the more abstract scale $\mathcal{E}$. 
For this mode the fields are 
\eqa{
    E_z &= 2 \sin\lp \pi x \rp  \sin\lp \pi y \rp \\
    B_x &=  i \sqrt{2} \sin\lp \pi x \rp  \cos\lp \pi y \rp \\
    B_y &= -i \sqrt{2} \cos\lp \pi x \rp  \sin\lp \pi y \rp 
}
thus the maximum surface magnetic field is $B_s = \sqrt{2} \mathcal{E}$. 
So write
\eqa{
\la{eq:Psig}
    \bar{P}_d 
   &=  \frac{Q_d \e^4 \mA^2 a^2 B_s^2 }{64 \pi^2 \omega^3 H} 
   \lvl G \rvl^2.
}




\section{Variations}

\subsection{Field Equations}

Now as a further check on \Eqs{eq:field-eq-F}{eq:field-eq-F'} we can also directly vary the Lagrangian in \Eq{eq:interaction-L}. 
This yields 
\eqa{
    \Lag + \delta\Lag = 
    - &\frac{1}{4} 
    \lp  F_{\mu\nu} + \delta F_{\mu\nu} \rp 
    \lp F^{\mu\nu} + \delta F^{\mu\nu} \rp 
     +  \frac{\e}{2}   
     \lp  F_{\mu\nu} + \delta F_{\mu\nu} \rp  
     \lp  F'^{\mu\nu} + \delta F'^{\mu\nu} \rp \\
     - &\frac{1}{4} 
    \lp  F'_{\mu\nu} + \delta F'_{\mu\nu} \rp 
    \lp F'^{\mu\nu} + \delta F'^{\mu\nu} \rp 
     + \frac{1}{2} \mA^2 
    \lp  A'_\mu + \delta A'_\mu \rp 
    \lp A'^\mu + \delta A'^\mu \rp \\
     -  &\lp  A_\mu + \delta A_\mu \rp  J_\tx{EM}^\mu. \\ 
    \approx 
    - &\frac{1}{4} 
    \lp  F_{\mu\nu} F^{\mu\nu} +2 F^{\mu\nu} \delta F_{\mu\nu} \rp 
    +  \frac{\e}{2}   
     \lp  F_{\mu\nu} F'^{\mu\nu} + F'^{\mu\nu} \delta F_{\mu\nu} 
     + F_{\mu\nu} \delta F'^{\mu\nu} \rp   \\
    - &\frac{1}{4} 
    \lp  F'_{\mu\nu} F'^{\mu\nu} +2 F'^{\mu\nu} \delta F'_{\mu\nu} \rp 
     + \frac{1}{2} \mA^2 
    \lp  A'_\mu A'^\mu + 2 A'^\mu \delta A'_\mu \rp 
     -  \lp  A_\mu + \delta A_\mu \rp  J_\tx{EM}^\mu \\
\Rightarrow \qq \delta\Lag =
    - &\frac{1}{2}  F^{\mu\nu} \delta F_{\mu\nu} 
    - \frac{1}{2} F'^{\mu\nu} \delta F'_{\mu\nu} 
    +  \frac{\e}{2}   
     \lp  F'^{\mu\nu} \delta F_{\mu\nu} +
      F_{\mu\nu} \delta F'^{\mu\nu} \rp  
     +  \mA^2 A'^\mu \delta A'_\mu 
     -   J_\tx{EM}^\mu \delta A_\mu  \\
     =- &\frac{1}{2} 
    \lp  F^{\mu\nu} - \e F'^{\mu\nu} \rp 
    \delta F_{\mu\nu} 
    -   J_\tx{EM}^\mu \delta A_\mu  
    - \frac{1}{2}
     \lp  F'^{\mu\nu} - \e  F^{\mu\nu} \rp 
     \delta F'_{\mu\nu} 
     +  \mA^2 A'^\mu \delta A'_\mu .
}
Now relate $\delta F$ to $\delta A$, 
\eqa{
    F_{\mu\nu} &= \partial_\mu A_\nu - \partial_\nu A_\mu \\
    F_{\mu\nu} + \delta F_{\mu\nu} &= 
    \partial_\mu \lp  A_\nu + \delta A_\nu \rp - 
    \partial_\nu \lp  A_\mu + \delta A_\mu \rp \\
    \delta F_{\mu\nu} &= 
    \partial_\mu \lp  \delta A_\nu \rp - 
    \partial_\nu \lp  \delta A_\mu \rp
}
and we can integrate by parts 
\eqa{
    \lp  F^{\mu\nu} - \e F'^{\mu\nu} \rp 
    \delta F_{\mu\nu} &= 
    \lp  F^{\mu\nu} - \e F'^{\mu\nu} \rp 
    \lp  
    \partial_\mu \lp  \delta A_\nu \rp - 
    \partial_\nu \lp  \delta A_\mu \rp \rp \\
    &=
    \lp  F^{\mu\nu} - \e F'^{\mu\nu} \rp 
    \partial_\mu \lp  \delta A_\nu \rp 
    - \lp  F^{\mu\nu} - \e F'^{\mu\nu} \rp 
    \partial_\nu \lp  \delta A_\mu \rp  \\
    &=
    \lp  F^{\mu\nu} - \e F'^{\mu\nu} \rp 
    \partial_\mu \lp  \delta A_\nu \rp 
    + \lp  F^{\nu\mu} - \e F'^{\nu\mu} \rp 
    \partial_\nu \lp  \delta A_\mu \rp  \\
    &=
    2 \lp F^{\mu\nu} - \e F'^{\mu\nu} \rp 
    \partial_\mu \lp  \delta A_\nu \rp \\
    &=
    - 2 \lp  \partial_\mu F^{\mu\nu} - \e \partial_\mu F'^{\mu\nu} 
    \rp \delta A_\nu + \lp \tx{boundary term}\rp \\
\Rightarrow \qq    - \frac{1}{2} 
    \lp  F^{\mu\nu} - \e F'^{\mu\nu} \rp 
    \delta F_{\mu\nu} &= 
     \lp \partial_\mu F^{\mu\nu} - \e \partial_\mu F'^{\mu\nu} 
    \rp \delta A_\nu 
}
and so $\delta \Lag$ is 
\eqa{
\delta \Lag  &= \lp \partial_\mu F^{\mu\nu} - 
     \e \partial_\mu F'^{\mu\nu} 
    \rp \delta A_\nu 
    -   J_\tx{EM}^\mu \delta A_\mu  
    + \lp \partial_\mu F'^{\mu\nu} - 
     \e \partial_\mu F^{\mu\nu} 
    \rp \delta A'_\nu 
     +  \mA^2 A'^\mu \delta A'_\mu  \\ 
    &= \lp \partial_\mu F^{\mu\nu} - 
     \e \partial_\mu F'^{\mu\nu} 
    -  J_\tx{EM}^\nu \rp \delta A_\nu  
    + \lp \partial_\mu F'^{\mu\nu} - 
     \e \partial_\mu F^{\mu\nu} 
     +  \mA^2 A'^\nu \rp \delta A'_\nu.
}
The field equations are 
\eqa{
\la{eq:field-eq-F-vary}
\partial_\mu F^{\mu\nu} - 
     \e \partial_\mu F'^{\mu\nu} 
    -   J_\tx{EM}^\nu &= 0 \\
\la{eq:field-eq-F'-vary}
    \partial_\mu F'^{\mu\nu} - 
     \e \partial_\mu F^{\mu\nu} 
     +  \mA^2 A'^\nu &= 0.
}
This agrees with \Eqs{eq:field-eq-F}{eq:field-eq-F'}. 

\subsection{Lorentz Force}

This Lorentz force must follow from the $A J$ coupling and the classical particle kinetic term,
\eqa{
    S = \int \diff{t} \lb - m \sqrt{1 - \dot{\vec{r}}^2}
    + \int \diffp{3}{x} A_\mu J^\mu \rb.
}
Here $\vec{r}(t)$ is the trajectory of the particle and $J$ is the current due to that trajectory. 
Taking the particle to have charge $e$,
\eqa{
    J^\mu &= e \delta^{(3)}\lb\vec{x} - \vec{r}(t)\rb
    \lp  1, \, \dot{\vec{r}}(t) \rp .
}
The EM current term in the action is  
\eqa{
S &= \int \diff{t} \diffp{3}{x} A_\mu 
\cdot \lp  1, \, \dot{\vec{r}}(t) \rp  \, 
  e \delta^{(3)}\lb\vec{x} - \vec{r}(t)\rb  \\
&= \int \diff{t} e A_\mu(r) \cdot \lp  1, \, \dot{\vec{r}}(t) \rp \\
&= \int \diff{t} e A_0(r) - e A_i(r) \dot{r}_i 
}
And so we have the classical action of a charge $e$ in a background field,
\eqa{
S &= \int \diff{t} \lb - m \sqrt{1 - \dot{\vec{r}}^2}
 + e A_0(r) - e A_i(r) \dot{r}_i  \rb
}
We now vary the trajectory: $\vec{r}(t) \rightarrow \vec{r}(t) + \delta \vec{r}(t)$. 
\eqa{
 \Lag &=  - m \sqrt{1 - 
  \partial_t (r_i + \delta r_i) \partial_t (r^i + \delta r^i)}
 + e A_0(r + \delta r) - e A_i(r + \delta r) \partial_t (r^i + \delta r^i) 
}

\bibliography{notes}

\end{document}